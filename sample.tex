% !TeX program = xelatex
% !TeX spellcheck = fa_IR

\documentclass{ui_proposal}

\usepackage[colorlinks=true]{hyperref}

\usepackage{amssymb}
\usepackage{graphicx}
\usepackage{listings}
\usepackage{color}
\usepackage{float}
\usepackage{hhline}
\usepackage{colortbl}
\usepackage{hyperref}
\usepackage[framemethod=TikZ]{mdframed}
\usepackage{lipsum}
\usepackage{setspace}
\usepackage{diagbox}
\usepackage[numbers, sort&compress]{natbib}

\usepackage{hypernat}
\usepackage{xcolor}

%در نسخه پرینت از جالت کامنت‌ دربیاورید تا لینک‌ها سیاه بوده و رنگی نباشند.
%\hypersetup{
%	citecolor=black,
%	linkcolor=black
%}

\usepackage[all]{hypcap}

\usepackage{xepersian}
\settextfont{XB Niloofar}
\setdigitfont{Parsi Digits}


\newcommand{\mylr}[1]{\texorpdfstring{\lr{#1}}{#1}}
\eqenvironment{نکات}{itemize}
\newtheorem{definition}{تعریف}
\eqenvironment{تعریف}{definition}
\eqcommand{مورد}{item}

\renewcommand{\descriptionlabel}[1]%
{\hspace{\labelsep} $\checkmark$ \hspace{\labelsep}\textbf{#1}}

\آرم{logo2.jpg}
\دانشجو{مهدی تیمورلوئی}
\شماره‌دانشجویی{901411702}
\ایمیل{m.teymourlouie@eng.ui.ac.ir}
\تلفن‌تماس{}
\گروه{کامپیوتر}
\دانشکده{فنی و مهندسی}
\‌رشته‌تحصیلی{مهندسی کامپیوتر}
\گرایش{نرم‌افزار}
\مقطع{دکتری}
\سنوات{۴.۵}

\‌استادراهنما{محمدعلی نعمت‌بخش}
\گروه‌راهنما{کامپیوتر}
\دانشگاه‌راهنما{دانشگاه اصفهان}
\مرتبه‌راهنما{دانشیار}
\تخصص‌راهنما{وب معنایی، داده‌کاوی}
\استخدام‌راهنما{رسمی}
\تعدادارشدراهنما{}
\تعداددکتری‌راهنما{}

%\استا‌دراهنمای‌دوم{احمد زائری}
%\گروه‌راهنمای‌دوم{کامپیوتر}
%\دانشگاه‌راهنمای‌دوم{دانشگاه اصفهان}
%\مرتبه‌راهنمای‌دوم{استادیار}
%\تخصص‌راهنمای‌دوم{وب معنایی}
%\استخدام‌راهنمای‌دوم{پیمانی}
%\تعدادارشدراهنمای‌دوم{}
%\تعداددکتری‌راهنمای‌دوم{}

%\استادمشاور{}
%\گروه‌مشاور{}
%\دانشگاه‌مشاور{}
%\مرتبه‌مشاور{}
%\تخصص‌مشاور{}
%\ملاحظات‌مشاور{}

%\استادمشاوردوم{}
%\گروه‌مشاوردوم{}
%\دانشگاه‌مشاوردوم{}
%\مرتبه‌مشاوردوم{}
%\تخصص‌مشاوردوم{}
%\ملاحظات‌مشاوردوم{}

\سایت{irandoc.ac.ir}
\شماره‌ثبت{}
\تاریخ‌ثبت{}

\کداولویت{9-7}
\عنوان‌اولویت{نظام‌های الکترونیکی (دولت،تجارت، سلامت و نظایر آن) و ارتقاء کمی و کیفی}
\حمایت‌کننده{}
\موضوع‌مورد‌حمایت{}
\عنوان‌فارسی {یک روش اشکال‌زدایی برای ارتقای درستی آنتولوژی}
\عنوان‌لاتین{A Debugging Approach to Improve Ontology Correctness}

%برای تیک خوردن هر کدام، کافی است از حالت کامنت خارج شود.
\کاربردی{1}
%\بنیادی{1}
\توسعه‌ای{1}

\کلیدواژه‌ها{ 
\شروع{شمارش}
\مورد اشکال‌زدایی (\lr{Debugging})
\مورد آنتولوژی (\lr{Ontology})
\مورد درستی (\lr{Correctness})
\مورد ناسازگاری (\lr{Incoherency})
\مورد تناقض (\lr{Incosistency})

\پایان{شمارش} }
\begin{document}
\maketitlepage

%%%%%%%%%%%%%%%%%%%%%%%%%%%%%%%%%%%%%%%%%%%%%%%%
\قسمت{شرح و بیان مسئله پژوهشی}
\برچسب{sec:problem}

%%%%%%%%%%%%%%%%%%%%%%%%%%%%%%%%%%%%%%%%%%%%%%%%
\قسمت{پیشینه و تاریخچه موضوع تحقیق} % مطالعات و تحقیقاتی که در رابطه با این موضوع صورت گرفته و نتایج حاصل از آن
\برچسب{sec:literature_review}

%%%%%%%%%%%%%%%%%%%%%%%%%%%%%%%%%%%%%%%%%%%%%%%%
\قسمت{اهداف تحقیق}
\برچسب{sec:goals}

%%%%%%%%%%%%%%%%%%%%%%%%%%%%%%%%%%%%%%%%%%%%%%%
\قسمت{اهمیت و ارزش تحقیق}
\برچسب{sec:importance}

%%%%%%%%%%%%%%%%%%%%%%%%%%%%%%%%%%%%%%%%%%%%%%%
\قسمت{کاربرد  نتایج تحقیق} % (رفع نیازهای ملی، ارائه نظریه جدید، عبور از مرزهای دانش، انتشارات علمی،ثبت اختراع، تولید محصول و تجاری سازی و ...)
\برچسب{sec:results}

%%%%%%%%%%%%%%%%%%%%%%%%%%%%%%%%%%%%%%%%%%%%%%%
\قسمت{فرضیه‌ها یا سوال‌های تحقیق}
\برچسب{sec:research_question}

%%%%%%%%%%%%%%%%%%%%%%%%%%%%%%%%%%%%%%%%%%%%%%%
\قسمت{روش تحقیق}
\برچسب{sec:methodology}

\زیرقسمت{نوع مطالعه و روش بررسی فرضیه‌ها و یا پاسخ‌گوئی به سوالات} %(توصیفی، تجربی، تحلیل محتوا، اسنادی، تاریخی و ...)} 
\برچسب{subsec:type}

\زیرقسمت{مراحل اجرایی تحقیق}
\برچسب{subsec:steps}

\زیرقسمت{جامعه آماری} % در صورت لزوم
\برچسب{subsec:population}

\زیرقسمت{روش و طرح نمونه‌برداری}
\برچسب{subsec:sampling}

\زیرقسمت{حجم نمونه و روش محاسبه}
\برچسب{subsec:sample_size}

\زیرقسمت{ابزار گردآوری داده‌ها} % (پرسش‌نامه، مصاحبه و ...)
\برچسب{subsec:data_tools}

\زیرقسمت{ابزار تجزیه و تحلیل}
\برچسب{subsec:analyze_tools}

%%%%%%%%%%%%%%%%%%%%%%%%%%%%%%%%%%%%%%%%%%%%%%%

\makebibliography{your_references}   % نام دیتابیس مراجع  شما با فرمت bibtex

%%%%%%%%%%%%%%%%%%%%%%%%%%%%%%%%%%%%%%%%%%%%%%%
\pagebreak
\restoregeometry
\newgeometry{includeheadfoot=false, marginparwidth=0cm, marginparsep=0cm, textwidth=160mm, headheight=0cm, headsep=0cm}%end of newgeometry
\قسمت{جدول زمانی و مراحل اجرا}
\برچسب{sec:schedule}
\begin{center}
\vspace{1cm}
\begin{tabular}{|m{1.5cm}|m{9cm}|m{2cm}|}
\hline
\grayrow \centering {\textbf{شماره}} & \centering {\textbf{فعالیت}}&\centering {\textbf{زمان (ماه)}} \tabularnewline
\hline
\وسط‌چین 1 & مطالعه روش‌های بررسی درستی & \وسط‌چین 4 \tabularnewline \hline
\وسط‌چین 2& ارائه روشی برای شناسایی و رفع گزاره‌های نادرست & \وسط‌چین  8 \tabularnewline \hline
\وسط‌چین 3& پیاده‌سازی در قالب یک افزونه‌ در  \lr{Prot\'eg\'e} & \وسط‌چین 2\tabularnewline \hline
\وسط‌چین  4& تهیه مجموعه داده آزمون & \وسط‌چین 4 \tabularnewline \hline
\وسط‌چین  5& آزمون و ارزیابی & \وسط‌چین 5 \tabularnewline \hline
\وسط‌چین  6& نگارش مقالات و انتشار آن‌ها & \وسط‌چین  12 \tabularnewline \hline
\وسط‌چین  7& نگارش پایان‌نامه، برگزاری دفاع و انجام اصلاحات & \وسط‌چین  7 \tabularnewline \hline
\end{tabular}
\end{center}
\vspace{1cm}
\begin{center}
\setlength{\tabcolsep}{2pt}
\def\markcell{\cellcolor[gray]{.5}}
\begin{tabular}{|m{2.2cm}|m{9pt}|m{9pt}|m{9pt}|m{9pt}|m{9pt}|m{9pt}|m{9pt}|m{9pt}|m{9pt}|m{9pt}|m{9pt}|m{9pt}|m{9pt}|m{9pt}|m{9pt}|m{9pt}|m{9pt}|m{9pt}|m{9pt}|m{9pt}|m{9pt}|m{9pt}|m{9pt}|m{9pt}|m{9pt}|m{9pt}|m{9pt}|m{9pt}|m{9pt}|m{9pt}|}
\hline
\multirow {2} {*} {\diagbox [dir=NE] {\small \bf{زمان}} {\small \bf{فعالیت}}}
& \multicolumn {6}{c|}{\graycell ۱۳۹۲}
 &\multicolumn {1}{c}{\graycell}
& \multicolumn {10}{c}{\graycell۱۳۹۳} &\multicolumn {1}{c|}{\graycell}
&\multicolumn {1}{c}{\graycell}
& \multicolumn {10}{c}{\graycell۱۳۹۴}
&\multicolumn {1}{c|}{\graycell} \\
\cline{2-31}
& \centering {\scriptsize {‌۷}} & \centering {\scriptsize {‌۸}} & \centering {\scriptsize {۹}} & \centering {\scriptsize {۱۰}} & \centering {\scriptsize {۱۱}} & \centering {\scriptsize {۱۲}}
& \centering {\scriptsize {‌۱}} &  \centering {\scriptsize {‌۲}} & \centering {\scriptsize {‌۳}} & \centering {\scriptsize {‌۴}} & \centering {\scriptsize {‌۵}} & \centering {\scriptsize {۶}} & \centering {\scriptsize {‌۷}} & \centering {\scriptsize {‌۸}} & \centering {\scriptsize  {۹}} & \centering {\scriptsize {۱۰}} & \centering {\scriptsize {۱۱}} & \centering {\scriptsize {۱۲}}
& \centering {\scriptsize {‌۱}} & \centering {\scriptsize {‌۲}} & \centering {\scriptsize {‌۳}} & \centering {\scriptsize {‌۴}} & \centering {\scriptsize {‌۵}} & \centering {\scriptsize {۶}}& \centering {\scriptsize {‌۷}} & \centering {\scriptsize {‌۸}} & \centering {\scriptsize {۹}} & \centering {\scriptsize {۱۰}} & \centering {\scriptsize  {۱۱}} & \centering {\scriptsize {۱۲}}
 \tabularnewline
\hline 
\tallrow \centering {۱}
& & &  \markcell& \markcell& \markcell& \markcell
& & & & & & & & & & & & 
& & & & & & & & & & & &
\tabularnewline \hline
\tallrow \centering {۲} 
&  & &  & & &\markcell 
\markcell&\markcell &\markcell &\markcell &\markcell &\markcell &\markcell &\markcell & & & & & 
& & & & & & & & & & & &
\tabularnewline \hline
\tallrow \centering {۳} 
&  & &  & & & 
& & & & & &\markcell &\markcell & & & & & 
& & & & & & & & & & & &
 \tabularnewline \hline
\tallrow \centering {۴} 
&  & &  & & & 
& & & & & & &\markcell &\markcell &\markcell &\markcell & & 
& & & & & & & & & & & &
\tabularnewline \hline
\tallrow \centering {۵} 
&  & &  & & & 
& & & & & & & &\markcell &\markcell &\markcell &\markcell & \markcell
& & & & & & & & & & & &
\tabularnewline \hline
\tallrow \centering {۶} 
&  & &  & & & 
& & & & & & & & & &\markcell &\markcell &\markcell 
&\markcell &\markcell &\markcell & \markcell&\markcell &\markcell &\markcell &\markcell &\markcell & & &
\tabularnewline \hline
\tallrow \centering {۷} 
&  & &  & & & 
& & & & & & & & & & & & 
& & & & &\markcell &\markcell &\markcell &\markcell &\markcell &\markcell & \markcell&
\tabularnewline \hline
\end{tabular}
\end{center}
\pagebreak
\restoregeometry
%%%%%%%%%%%%%%%%%%%%%%%%%%%%%%%%%%%%%%%%%%%%%%%
\قسمت{هزینه‌های ضروری درخواستی} % مبلغ و شرح کامل آن پیوست گردد.
\برچسب{sec:costs}
\begin{center}
\begin{tabular}{|m{1.5cm}|m{5cm}|m{2.5cm}|m{2.5cm}|m{2.5cm}|}
\hline
\grayrow
\tallrow \centering {\textbf{ردیف}} & \centering {\textbf{وسایل و تجهیزات }}&\centering {\textbf{تعداد}} & \centering {\textbf{برآورد هزینه}} & \centering {\textbf{جمع}} \tabularnewline
\hline
\وسط‌چین  1& رایانه‌ با قدرت پردازشی مناسب و با دارا بودن حداقل 24 گیگابایت رم & \وسط‌چین  1 & 25000000& 25000000 \tabularnewline
 & & & & \tabularnewline
 & & & & \tabularnewline
 & & & & \tabularnewline \hline
\end{tabular}
\end{center}
\vspace{.10cm}
\begin{center}
\begin{tabular}{|m{1.5cm}|m{5cm}|m{2.5cm}|m{2.5cm}|m{2.5cm}|}
\hline
\grayrow
\tallrow \centering {\textbf{ردیف}} & \centering {\textbf{مواد مصرفی }}&\centering {\textbf{تعداد}} & \centering {\textbf{برآورد هزینه}} & \centering {\textbf{جمع}} \tabularnewline
\hline
 & & & & \tabularnewline
 & & & & \tabularnewline
 & & & & \tabularnewline
 & & & & \tabularnewline
 & & & & \tabularnewline \hline
\end{tabular}
\end{center}
\vspace{.10cm}
\begin{center}
\begin{tabular}{|m{1.5cm}|m{5cm}|m{2.5cm}|m{2.5cm}|m{2.5cm}|}
\hline
\grayrow
\tallrow \centering {\textbf{ردیف}} & \centering {\textbf{سایر موارد }}&\centering {\textbf{تعداد}} & \centering {\textbf{برآورد هزینه}} & \centering {\textbf{جمع}} \tabularnewline
\hline
\وسط‌چین 1 & چاپ مقالات و شرکت در کنفرانس‌ &\وسط‌چین  3& 4000000& 15000000 \tabularnewline
\وسط‌چین 2 &  چاپ و تکثیر پایان‌نامه &\وسط‌چین  5 &  500000& 2500000 \tabularnewline
 & & & & \tabularnewline
 & & & & \tabularnewline
 & & & & \tabularnewline \hline
\end{tabular}
\end{center}
\vspace{0.10cm}
\begin{center}
\framebox [15.5cm] [r]{
\textbf {جمع کل هزینه‌های پیشنهادی:} 
42500000 ریال
\hspace{5mm}
معادل چهار میلیون و دویست و پنجاه هزار تومان
}
\end{center}
\pagebreak
\restoregeometry
%%%%%%%%%%%%%%%%%%%%%%%%%%%%%%%%%%%%%%%%%%%%%%%
\makesignaturepage
\makeethicspage
\end{document}