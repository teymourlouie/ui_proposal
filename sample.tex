% !TeX program = xelatex
% !TeX spellcheck = fa_IR

\documentclass{ui_proposal}

\usepackage[colorlinks=true]{hyperref}

\usepackage{amssymb}
\usepackage{graphicx}
\usepackage{listings}
\usepackage{color}
\usepackage{float}
\usepackage{hyperref}
\usepackage{hhline}
\usepackage{colortbl}
\usepackage[framemethod=TikZ]{mdframed}
\usepackage{lipsum}
\usepackage{setspace}
\usepackage{diagbox}
\usepackage[numbers, sort&compress]{natbib}
\usepackage{hypernat}
\usepackage[all]{hypcap}
%\usepackage{showframe}
\usepackage{xepersian}
\settextfont{XB Niloofar}
\setiranicfont{XB Zar Oblique}
\setdigitfont{Parsi Digits}


\newcommand{\mylr}[1]{\texorpdfstring{\lr{#1}}{#1}}

\آرم{logo2.jpg}
\دانشجو{مهدی تیمورلوئی}
\شماره‌دانشجویی{901411702}
\ایمیل{m.teymourlouie@eng.ui.ac.ir}
\تلفن‌تماس{09143012589}
\گروه{کامپیوتر}
\دانشکده{فنی و مهندسی}
\‌رشته‌تحصیلی{مهندسی کامپیوتر}
\گرایش{نرم‌افزار}
\مقطع{دکتری}
\سنوات{۴.۵}

\‌استادراهنما{محمدعلی نعمت‌بخش}
\گروه‌راهنما{کامپیوتر}
\دانشگاه‌راهنما{دانشگاه اصفهان}
\مرتبه‌راهنما{دانشیار}
\تخصص‌راهنما{}
\استخدام‌راهنما{رسمی}
\تعدادارشدراهنما{}
\تعداددکتری‌راهنما{}

\استا‌دراهنمای‌دوم{احمد زائری}
\گروه‌راهنمای‌دوم{کامپیوتر}
\دانشگاه‌راهنمای‌دوم{دانشگاه اصفهان}
\مرتبه‌راهنمای‌دوم{استادیار}
\تخصص‌راهنمای‌دوم{}
\استخدام‌راهنمای‌دوم{پیمانی}
\تعدادارشدراهنمای‌دوم{}
\تعداددکتری‌راهنمای‌دوم{}

%\استادمشاور{}
\گروه‌مشاور{کامپیوتر}
\دانشگاه‌مشاور{}
\مرتبه‌مشاور{}
\تخصص‌مشاور{}
\ملاحظات‌مشاور{}

\استادمشاوردوم{}
\گروه‌مشاوردوم{}
\دانشگاه‌مشاوردوم{}
\مرتبه‌مشاوردوم{}
\تخصص‌مشاوردوم{}
\ملاحظات‌مشاوردوم{}

\سایت{irandoc.ac.ir}
\شماره‌ثبت{۴۵۶۸۹۹۰}
\تاریخ‌ثبت{۱۳۹۲/۰۷/۲۰}

\کداولویت{۸-۷}
\عنوان‌اولویت{وب معنایی خیلی خوب است}
\حمایت‌کننده{گوگل}
\موضوع‌مورد‌حمایت{-}

\عنوان‌فارسی {اشکال‌زدایی از آنتولوژی‌ها با تأکید بر حذف ناسازگاری}
\عنوان‌لاتین{Ontology Debugging}

\کاربردی{1}
%\بنیادی{1}
\توسعه‌ای{1}

\کلیدواژه‌ها{
\begin{enumerate}
\item اشکال‌زدایی
\item آنتولوژی
\item انسجام
\item تناقض
\end{enumerate}
}

\begin{document}
\maketitlepage

%%%%%%%%%%%%%%%%%%%%%%%%%%%%%%%%%%%%%%%%%%%%%%%%
\قسمت{شرح و بیان مسئله پژوهشی}
\برچسب{sec:problem}

%%%%%%%%%%%%%%%%%%%%%%%%%%%%%%%%%%%%%%%%%%%%%%%%
\قسمت{پیشینه و تاریخچه موضوع تحقیق} % مطالعات و تحقیقاتی که در رابطه با این موضوع صورت گرفته و نتایج حاصل از آن
\برچسب{sec:literature_review}

%%%%%%%%%%%%%%%%%%%%%%%%%%%%%%%%%%%%%%%%%%%%%%%%
\قسمت{اهداف تحقیق}
\برچسب{sec:goals}

%%%%%%%%%%%%%%%%%%%%%%%%%%%%%%%%%%%%%%%%%%%%%%%
\قسمت{اهمیت و ارزش تحقیق}
\برچسب{sec:importance}

%%%%%%%%%%%%%%%%%%%%%%%%%%%%%%%%%%%%%%%%%%%%%%%
\قسمت{کاربرد  و نتایج تحقیق} % (رفع نیازهای ملی، ارائه نظریه جدید، عبور از مرزهای دانش، انتشارات علمی،ثبت اختراع، تولید محصول و تجاری سازی و ...)
\برچسب{sec:results}

%%%%%%%%%%%%%%%%%%%%%%%%%%%%%%%%%%%%%%%%%%%%%%%
\قسمت{فرضیه‌ها یا سوال‌های تحقیق}
\برچسب{sec:research_question}

%%%%%%%%%%%%%%%%%%%%%%%%%%%%%%%%%%%%%%%%%%%%%%%
\قسمت{روش تحقیق}
\برچسب{sec:methodology}

\زیرقسمت{نوع مطالعه و روش بررسی فرضیه‌ها و یا پاسخ‌گوئی به سوالات} %(توصیفی، تجربی، تحلیل محتوا، اسنادی، تاریخی و ...)} 
\برچسب{subsec:type}

\زیرقسمت{مراحل اجرایی تحقیق}
\برچسب{subsec:steps}

\زیرقسمت{جامعه آماری} % در صورت لزوم
\برچسب{subsec:population}

\زیرقسمت{روش و طرح نمونه‌برداری}
\برچسب{subsec:sampling}

\زیرقسمت{حجم نمونه و روش محاسبه}
\برچسب{subsec:sample_size}

\زیرقسمت{ابزار گردآوری داده‌ها} % (پرسش‌نامه، مصاحبه و ...)
\برچسب{subsec:data_tools}

\زیرقسمت{ابزار تجزیه و تحلیل}
\برچسب{subsec:analyze_tools}

%%%%%%%%%%%%%%%%%%%%%%%%%%%%%%%%%%%%%%%%%%%%%%%
{
\bibliographystyle{unsrtnat}
\latin
\renewcommand{\bibname}{\rl{{مراجع}\hfill}}
\bibliography{references.bib}
}
%%%%%%%%%%%%%%%%%%%%%%%%%%%%%%%%%%%%%%%%%%%%%%%
\قسمت{جدول زمانی و مراحل اجرا}
\برچسب{sec:schedule}
\pagebreak
\newgeometry{ignoreall=true, includeheadfoot=false, marginparwidth=0cm, marginparsep=0cm, textwidth=200mm, headheight=0cm, headsep=0cm, top=3mm, bottom=5mm, right=0.5cm, left=0.5cm}%end of newgeometry

\textbf{جدول زمانی و مراحل اجرا} \\
\begin{center}
\begin{tabular}{|m{1.5cm}|m{9cm}|m{2cm}|}
\hline
 \centering {\textbf{شماره}} & \centering {\textbf{فعالیت}}&\centering {\textbf{زمان (ماه)}} \tabularnewline
\hline
& & \tabularnewline \hline
& & \tabularnewline \hline
& & \tabularnewline \hline
& & \tabularnewline \hline
& & \tabularnewline \hline
& & \tabularnewline \hline
& & \tabularnewline \hline
\end{tabular}
\end{center}
\vspace{1cm}
\begin{center}
\setlength{\tabcolsep}{2pt}
\begin{tabular}{|m{2.2cm}|m{9pt}|m{9pt}|m{9pt}|m{9pt}|m{9pt}|m{9pt}|m{9pt}|m{9pt}|m{9pt}|m{9pt}|m{9pt}|m{9pt}|m{9pt}|m{9pt}|m{9pt}|m{9pt}|m{9pt}|m{9pt}|m{9pt}|m{9pt}|m{9pt}|m{9pt}|m{9pt}|m{9pt}|m{9pt}|m{9pt}|m{9pt}|m{9pt}|m{9pt}|m{9pt}|m{9pt}|m{9pt}|m{9pt}|m{9pt}|m{9pt}|m{9pt}|}
\hline
\multirow {2} {*} { \diagbox [dir=NE] {\small \bf{زمان}} {\small \bf{فعالیت}}} & 
\multicolumn {1}{c}{}
& \multicolumn {10}{c}{۱۳۹۲}&\multicolumn {1}{c|}{}
 &\multicolumn {1}{c}{}
& \multicolumn {10}{c}{۱۳۹۳} &\multicolumn {1}{c|}{}
&\multicolumn {1}{c}{}
& \multicolumn {10}{c}{۱۳۹۴}
&\multicolumn {1}{c|}{} \\
\cline{2-37}
& \centering {\scriptsize {‌۱}} & \centering {\scriptsize {‌۲}} & \centering {\scriptsize {‌۳}} & \centering {\scriptsize {‌۴}} & \centering {\scriptsize {‌۵}} & \centering {\scriptsize {۶}} & \centering {\scriptsize {‌۷}} & \centering {\scriptsize {‌۸}} & \centering {\scriptsize {۹}} & \centering {\scriptsize {۱۰}} & \centering {\scriptsize {۱۱}} & \centering {\scriptsize {۱۲}}
& \centering {\scriptsize {‌۱}} &  \centering {\scriptsize {‌۲}} & \centering {\scriptsize {‌۳}} & \centering {\scriptsize {‌۴}} & \centering {\scriptsize {‌۵}} & \centering {\scriptsize {۶}} & \centering {\scriptsize {‌۷}} & \centering {\scriptsize {‌۸}} & \centering {\scriptsize  {۹}} & \centering {\scriptsize {۱۰}} & \centering {\scriptsize {۱۱}} & \centering {\scriptsize {۱۲}}
& \centering {\scriptsize {‌۱}} & \centering {\scriptsize {‌۲}} & \centering {\scriptsize {‌۳}} & \centering {\scriptsize {‌۴}} & \centering {\scriptsize {‌۵}} & \centering {\scriptsize {۶}}& \centering {\scriptsize {‌۷}} & \centering {\scriptsize {‌۸}} & \centering {\scriptsize {۹}} & \centering {\scriptsize {۱۰}} & \centering {\scriptsize  {۱۱}} & \centering {\scriptsize {۱۲}}
 \tabularnewline
\hline 
\tallrow \centering {۱}
&&&&&&&&&&&&
&&&&&&&&&&&&
&&&&&&&&&&&&
\tabularnewline \hline
\tallrow \centering {۲} 
&&&&&&&&&&&&
&&&&&&&&&&&&
&&&&&&&&&&&&
\tabularnewline \hline
\tallrow \centering {۳} 
&&&&&&&&&&&&
&&&&&&&&&&&&
&&&&&&&&&&&&
 \tabularnewline \hline
\tallrow \centering {۴} 
&&&&&&&&&&&&
&&&&&&&&&&&&
&&&&&&&&&&&&
\tabularnewline \hline
\tallrow \centering {۵} 
&&&&&&&&&&&&
&&&&&&&&&&&&
&&&&&&&&&&&&
\tabularnewline \hline
\tallrow \centering {۶} 
&&&&&&&&&&&&
&&&&&&&&&&&&
&&&&&&&&&&&&
\tabularnewline \hline
\tallrow \centering {۷} 
&&&&&&&&&&&&
&&&&&&&&&&&&
&&&&&&&&&&&&
\tabularnewline \hline
\end{tabular}
\end{center}
\pagebreak
\restoregeometry
%%%%%%%%%%%%%%%%%%%%%%%%%%%%%%%%%%%%%%%%%%%%%%%
\قسمت{هزینه‌های ضروری درخواستی} % مبلغ و شرح کامل آن پیوست گردد.
\برچسب{sec:costs}

%\newgeometry{ignoreall=true, includeheadfoot=false, marginparwidth=0cm, marginparsep=0cm, textwidth=200mm, headheight=0cm, headsep=0cm, top=3mm, bottom=5mm, right=0.5cm, left=0.5cm}%end of newgeometry

%\begin{mdframed}[roundcorner=20pt, innerrightmargin =3mm, innerleftmargin=3mm, linewidth=0.4pt, outerlinewidth=0.4pt, middlelinewidth=0.4pt, frametitle = {\textbf{هزینه‌های ضروری درخواستی: (مبلغ و شرح کامل آن) پیوست گردد.}}, skipabove=1cm, skipbelow =1cm, rightmargin=1cm, leftmargin=1cm]
%{
\begin{center}
\begin{tabular}{|m{1.5cm}|m{5cm}|m{2.5cm}|m{2.5cm}|m{2.5cm}|}
\hline
\grayrow
\tallrow \centering {\textbf{ردیف}} & \centering {\textbf{وسایل و تجهیزات }}&\centering {\textbf{تعداد}} & \centering {\textbf{برآورد هزینه}} & \centering {\textbf{جمع}} \tabularnewline
\hline
 & & & & \tabularnewline
 & & & & \tabularnewline
 & & & & \tabularnewline
 & & & & \tabularnewline
 & & & & \tabularnewline \hline
\end{tabular}
\end{center}
\vspace{1cm}
\begin{center}
\begin{tabular}{|m{1.5cm}|m{5cm}|m{2.5cm}|m{2.5cm}|m{2.5cm}|}
\hline
\grayrow
\tallrow \centering {\textbf{ردیف}} & \centering {\textbf{مواد مصرفی }}&\centering {\textbf{تعداد}} & \centering {\textbf{برآورد هزینه}} & \centering {\textbf{جمع}} \tabularnewline
\hline
 & & & & \tabularnewline
 & & & & \tabularnewline
 & & & & \tabularnewline
 & & & & \tabularnewline
 & & & & \tabularnewline \hline
\end{tabular}
\end{center}
\vspace{1cm}
\begin{center}
\begin{tabular}{|m{1.5cm}|m{5cm}|m{2.5cm}|m{2.5cm}|m{2.5cm}|}
\hline
\grayrow
\tallrow \centering {\textbf{ردیف}} & \centering {\textbf{سایر موارد }}&\centering {\textbf{تعداد}} & \centering {\textbf{برآورد هزینه}} & \centering {\textbf{جمع}} \tabularnewline
\hline
 & & & & \tabularnewline
 & & & & \tabularnewline
 & & & & \tabularnewline
 & & & & \tabularnewline
 & & & & \tabularnewline \hline
\end{tabular}
\end{center}
\vspace{0.5cm}
\begin{center}
\framebox [14cm] [r]{
\textbf {جمع کل هزینه‌های پیشنهادی:}
} \\
\end{center}
%}
%\end{mdframed}
\pagebreak
\restoregeometry
%%%%%%%%%%%%%%%%%%%%%%%%%%%%%%%%%%%%%%%%%%%%%%%
\makesignaturepage
\makeethicspage


\end{document}